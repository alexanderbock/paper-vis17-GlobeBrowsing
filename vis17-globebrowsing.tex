\documentclass[journal]{vgtc}                % final (journal style)
%\documentclass[review,journal]{vgtc}         % review (journal style)
%\documentclass[widereview]{vgtc}             % wide-spaced review
%\documentclass[preprint,journal]{vgtc}       % preprint (journal style)

%% Uncomment one of the lines above depending on where your paper is
%% in the conference process. ``review'' and ``widereview'' are for review
%% submission, ``preprint'' is for pre-publication, and the final version
%% doesn't use a specific qualifier.

%% Please use one of the ``review'' options in combination with the
%% assigned online id (see below) ONLY if your paper uses a double blind
%% review process. Some conferences, like IEEE Vis and InfoVis, have NOT
%% in the past.

%% Please note that the use of figures other than the optional teaser is not permitted on the first page
%% of the journal version.  Figures should begin on the second page and be
%% in CMYK or Grey scale format, otherwise, colour shifting may occur
%% during the printing process.  Papers submitted with figures other than the optional teaser on the
%% first page will be refused. Also, the teaser figure should only have the
%% width of the abstract as the template enforces it.

%% These few lines make a distinction between latex and pdflatex calls and they
%% bring in essential packages for graphics and font handling.
%% Note that due to the \DeclareGraphicsExtensions{} call it is no longer necessary
%% to provide the the path and extension of a graphics file:
%% \includegraphics{diamondrule} is completely sufficient.
%%
\ifpdf%                                % if we use pdflatex
  \pdfoutput=1\relax                   % create PDFs from pdfLaTeX
  \pdfcompresslevel=9                  % PDF Compression
  \pdfoptionpdfminorversion=7          % create PDF 1.7
  \ExecuteOptions{pdftex}
  \usepackage{graphicx}                % allow us to embed graphics files
  \DeclareGraphicsExtensions{.pdf,.png,.jpg,.jpeg} % for pdflatex we expect .pdf, .png, or .jpg files
\else%                                 % else we use pure latex
  \ExecuteOptions{dvips}
  \usepackage{graphicx}                % allow us to embed graphics files
  \DeclareGraphicsExtensions{.eps}     % for pure latex we expect eps files
\fi%

%% it is recomended to use ``\autoref{sec:bla}'' instead of ``Fig.~\ref{sec:bla}''
\graphicspath{{figures/}{pictures/}{images/}{./}} % where to search for the images

\usepackage{microtype}                 % use micro-typography (slightly more compact, better to read)
\PassOptionsToPackage{warn}{textcomp}  % to address font issues with \textrightarrow
\usepackage{textcomp}                  % use better special symbols
\usepackage{mathptmx}                  % use matching math font
\usepackage{times}                     % we use Times as the main font
\renewcommand*\ttdefault{txtt}         % a nicer typewriter font
\usepackage{cite}                      % needed to automatically sort the references
%\usepackage{color}
\usepackage{xcolor} % Some more colors not defined in "color" package
% \usepackage{tabu}                      % only used for the table example
% \usepackage{booktabs}                  % only used for the table example
\usepackage{todonotes}



\newcommand{\kallecomment}[1]{\textbf{[-Kalle-~}
    \textcolor{orange}{#1}
    \textbf{~]}}

\newcommand{\emilcomment}[1]{\textbf{[-Emil-~}
    \textcolor{red}{#1}
    \textbf{~]}}

\newcommand{\alexcomment}[1]{\textbf{[-Alex-~}
    \textcolor{magenta}{#1}
    \textbf{~]}}

\newcommand{\anderscomment}[1]{\textbf{[-Anders-~}
    \textcolor{cyan}{#1}
    \textbf{~]}}

\newcommand{\etal}{\emph{et~al.}}


%% We encourage the use of mathptmx for consistent usage of times font
%% throughout the proceedings. However, if you encounter conflicts
%% with other math-related packages, you may want to disable it.

%% In preprint mode you may define your own headline.
%\preprinttext{To appear in IEEE Transactions on Visualization and Computer Graphics.}

%% If you are submitting a paper to a conference for review with a double
%% blind reviewing process, please replace the value ``0'' below with your
%% OnlineID. Otherwise, you may safely leave it at ``0''.
\onlineid{0}

%% declare the category of your paper, only shown in review mode
\vgtccategory{Research}
%% please declare the paper type of your paper to help reviewers, only shown in review mode
%% choices:
%% * algorithm/technique
%% * application/design study
%% * evaluation
%% * system
%% * theory/model
\vgtcpapertype{System}

%% Paper title.
\title{Marsbrowsing}

%% This is how authors are specified in the journal style

%% indicate IEEE Member or Student Member in form indicated below
\author{Roy G. Biv, Ed Grimley, \textit{Member, IEEE}, and Martha Stewart}
\authorfooter{
%% insert punctuation at end of each item
\item
 Roy G. Biv is with Starbucks Research. E-mail: roy.g.biv@aol.com.
\item
 Ed Grimley is with Grimley Widgets, Inc.. E-mail: ed.grimley@aol.com.
\item
 Martha Stewart is with Martha Stewart Enterprises at Microsoft
 Research. E-mail: martha.stewart@marthastewart.com.
}

%other entries to be set up for journal
\shortauthortitle{Biv \MakeLowercase{\textit{et al.}}: Global Illumination for Fun and Profit}
%\shortauthortitle{Firstauthor \MakeLowercase{\textit{et al.}}: Paper Title}

%% Abstract section.
\abstract{

The mapping of globes, (planet, moons, and other nearly elliptical celestial objects) within our solar system is carried out by scientists of space organizations. The gathered image data are used in planning of future space missions but seldom makes its way out to the general public even though NASA provides a lot of it openly.

Much has been done in the field of terrain and globe rendering and current technology relies on techniques such as out-of-core, and dynamic level of detail rendering, multi threaded data acquisition and consideration of precision limitations.

These systems are often either specialized for researchers where accuracy is the main concern, or for games, where the user experience plays a major role.

We present an application for visualizing the same data that space scientists analyze in the context of a virtual environment representing the space that we are exploring. Given the contextualization that the real space provides, we can model the solar system using positional data for modelling the planets' orbits together with the space probes that explores them.

The main focus of this paper is to describe the combination of different datasets gathered for mapping out planet surfaces features, particularly for Mars due to the extensive amount of map data gathered by the various missions on this planet.

\kallecomment{Some more on the methods then results and conclusions. Please bash my text and make it better :). Are we focusing on the right things for the abstract?}

\begin{enumerate}

\item Results
\item Conclusions
\end{enumerate}
%
} % end of abstract

%% Keywords that describe your work. Will show as 'Index Terms' in journal
%% please capitalize first letter and insert punctuation after last keyword
\keywords{Radiosity, global illumination, constant time}

%% ACM Computing Classification System (CCS). 
%% See <http://www.acm.org/class/1998/> for details.
%% The ``\CCScat'' command takes four arguments.

\CCScatlist{ % not used in journal version
 \CCScat{K.6.1}{Management of Computing and Information Systems}%
{Project and People Management}{Life Cycle};
 \CCScat{K.7.m}{The Computing Profession}{Miscellaneous}{Ethics}
}

%% Uncomment below to include a teaser figure.
\teaser{
  \centering
  %\includegraphics[width=\linewidth]{CypressView}
  %\caption{In the Clouds: Vancouver from Cypress Mountain. Note that the teaser may not be wider than the abstract block.}
%	\label{fig:teaser}
}

%% Uncomment below to disable the manuscript note
%\renewcommand{\manuscriptnotetxt}{}

%% Copyright space is enabled by default as required by guidelines.
%% It is disabled by the 'review' option or via the following command:
% \nocopyrightspace

\vgtcinsertpkg

%%%%%%%%%%%%%%%%%%%%%%%%%%%%%%%%%%%%%%%%%%%%%%%%%%%%%%%%%%%%%%%%
%%%%%%%%%%%%%%%%%%%%%% START OF THE PAPER %%%%%%%%%%%%%%%%%%%%%%
%%%%%%%%%%%%%%%%%%%%%%%%%%%%%%%%%%%%%%%%%%%%%%%%%%%%%%%%%%%%%%%%%

\begin{document}

%% The ``\maketitle'' command must be the first command after the
%% ``\begin{document}'' command. It prepares and prints the title block.

%% the only exception to this rule is the \firstsection command
\firstsection{Introduction} \label{sec:introduction}
\maketitle
\kallecomment{Maybe some more catchy name of the article? "Browsing the Red Planet", "Browsing Mars", "Unveiling Mars"? :P}

\begin{enumerate}
\item Visualizing space data is important because its expensive

\kallecomment{What is expensive?}

\item There exists a vast amount of data from Mars orbiters

The amount of data currently available from various space missions is extensive. NASA alone offers more than 100 TB of images from various planetary space missions through the Planetary Data System, which is available as open data.

\kallecomment{\url{https://open.nasa.gov/blog/what-is-nasa-doing-with-big-data-today/}}

NASA's Viking program, launched in the year 1975, gathered important information about Mars and its surface features from the two orbiting satellites and the landers put on the surface of the planet. Today, the most important large scale imaging campaign is carried out by the Mars Reconnaissance Orbiter (MRO). This satellite carries the MARCI (Mars Color Imager), CTX (Context Camera) and HiRISE (High Resolution Imaging Science Experiment) cameras which are used to image the surface at different resolutions.

\kallecomment{\url{http://pds-imaging.jpl.nasa.gov/portal/}}

Our goal is to, using visualization, bring space science to the public as well as providing tools for scientists to talk about space research. Part of this relies on the ability to visualize planetary science by providing an accurate representation the globes in our solar system as well as the ability to go there virtually, using the open data available from various space missions.

\item Missing spatial understanding from looking at pure images

\kallecomment{any fancy visualization terms that I don't know about which can be used?}

Raw image data, gathered from space missions, is easily visualized flat on screen. However, since ... 

\item Contextualization of scientific data (being able to show satellites in the same context as surface features), (domes, vr headsets, etc)

Using the tools that scientific data visualization allows, it is possible to get a better understanding of the space missions. By allowing contol of time flow, showing satellites and space probes in the same context as the planets, and doing this using immersive display systems gives intuitive understanding of the space around us and how scientists gather knowledge about it.

Typical examples of use cases for interactive visualization are large scale dome presentations of real time presentations, virtual reality headsets, and touch tables; where the audience can experience flight through space and time, experiencing what we currently know about the cosmos.

\item Stereoscopic reconstruction from multiple image passes

\kallecomment{Maybe not suitable to put here already?}

\item What datasets are available (Viking, MOLA, CTX, HiRISE) and whats their resolutions

\kallecomment{MDIM 2.0 (references available): \url{https://astrogeology.usgs.gov/maps/mdim-2-1}}

\kallecomment{Colorized Viking global: \url{https://astrogeology.usgs.gov/search/map/Mars/Viking/MDIM21/Mars_Viking_MDIM21_ClrMosaic_global_232m}}

\kallecomment{MGS MOLA (references available): \url{https://astrogeology.usgs.gov/search/map/Mars/GlobalSurveyor/MOLA/Mars_MGS_MOLA_DEM_mosaic_global_463m}}

\kallecomment{MRO HiRISE: \url{https://astrogeology.usgs.gov/maps/mars-hirise-camera and https://astrogeology.usgs.gov/missions/mars-reconnaissance-orbiter}}

The Mars Global Digital Image Mosaic (MDIM) is a global image dataset with a resolution of 256 pixels/degree. This dataset is put together from images from the Viking Orbiter imaging system. The latest version of this image mosaic, MDIM 2.1 uses ...

\kallecomment{Mention latest coordinate system for Mars, IAU/IAG 2000 adoption.. Use of MOLA for correction etc.}

The highest resolution global color mosaic available for Mars today is still composed of images from the Viking missions. The latest version, compiled by NASA AMES, is warped to match the latest grayscale MDIM 2.1 mosaic. 

The Mars Orbiter Laser Altimeter (MOLA) is an instrument on the Mars Global Surveyor (MGS) spacecraft. The digital elevation model (DEM) assembled from MOLA data maps each position on the globe with an offset from the Areoid, Mars' reference ellipsoid, to an average accuracy of +-3 m (ref). The dataset has a resolution of 463.0836 meters/pixel.

\kallecomment{+-3 m sounds too good to be true? Should check the reference.}

\kallecomment{Areoid: geoid but for Mars. Areography, Martian geography.}




\kallecomment{Other datasets? Does ESA have Mars missions of interest?}

\item How can this be applied to other planets

\kallecomment{Applied as in how to build datasets or as in how to render it? if the latter, it shouldn't be in introduction.}

\item A system for enabling future research thati s correctly contextualized
\item What is the science question // What is the point of this
\end{enumerate}
Length: About 1 page

\section{Related Work} \label{sec:relatedwork}

Global Information Systems (GIS) relies heavily on the ability to gather, transform, and visualize data with geospatial information. Maps and DEM's are typical examples of such data and research and development of software solutions to handle different parts of GIS, such as rendering, have lead to different applications in space visualization.

A globe rendering system needs to handle out of core rendering and level of detail management to avoid flooding the data caches and rendering times. 

\begin{enumerate}
\item the book

\kallecomment{3D Engine Design for Virtual Globes?}

\item terrain renderer
\item 3d reconstruction from images (stereoscopic and structure-from-motion)
\item GDAL
\item ``virtual presence'' systems
\item What else?
\end{enumerate}
Length: About 1 page\\
Note:  The page limit was increased to 9+2 pages this year (= 9 pages of manuscript, 2 pages of references). So we should make use of this and cite the hell out of everything that's related

\section{Overview} \label{sec:overview}
\begin{enumerate}
  \item What are the steps to get from a satellite to 3d terrain rendering
\begin{enumerate}
  \item Acquision (MRO information)
  \item Processing (AMES Stereo pipeline, ..., GDAL)
  \item Rendering (Globebrowsing)
\end{enumerate}
  \item short descriptions for each
\end{enumerate}
Length: About 2 pages

\section{Image Acquisition and Processing} \label{sec:imageacquisitionprocessing}
\begin{enumerate}
  \item MRO information, different resolution levels
  \item What are the available data products
  \item Ames stereo pipeline
  \item GDAL preprocessing
\end{enumerate}
Length: About 1-1.5 pages

\section{Rendering System} \label{sec:renderingsystem}
\begin{enumerate}
  \item All the steps to get from GDAL to a rendering on the screen
  
  \kallecomment{Tile pipeline as described in our thesis (but shorter).}
  
  \kallecomment{Abstraction layers.}
  
  \kallecomment{Rendering chunks. Shader implementation on a high level.}
  
  \kallecomment{Atmosphere?!?!!?!}
  
  \item Stereoscopic rendering
  \item Dome rendering
  \item Different resolution levels
  \item Rendering rover locations
\end{enumerate}
Length: About 2-2.5 pages (fill as much as the page limit (9+2) allows)

\section{Conclusion} \label{sec:system}
\begin{enumerate}
  \item Blabla; introduction in reverse
  \item Future work:
  \begin{enumerate}
    \item Focus more on scientific rather than engineering goals
  \end{enumerate}
\end{enumerate}
Length: About 1 page

%% if specified like this the section will be committed in review mode
\acknowledgments{
The authors wish to thank A, B, C. This work was supported in part by
a grant from XYZ.}

%\bibliographystyle{abbrv}
\bibliographystyle{abbrv-doi}
%\bibliographystyle{abbrv-doi-narrow}
%\bibliographystyle{abbrv-doi-hyperref}
%\bibliographystyle{abbrv-doi-hyperref-narrow}

\bibliography{references}
\end{document}

