\documentclass[journal]{vgtc}                % final (journal style)
%\documentclass[review,journal]{vgtc}         % review (journal style)
%\documentclass[widereview]{vgtc}             % wide-spaced review
%\documentclass[preprint,journal]{vgtc}       % preprint (journal style)

%% Uncomment one of the lines above depending on where your paper is
%% in the conference process. ``review'' and ``widereview'' are for review
%% submission, ``preprint'' is for pre-publication, and the final version
%% doesn't use a specific qualifier.

%% Please use one of the ``review'' options in combination with the
%% assigned online id (see below) ONLY if your paper uses a double blind
%% review process. Some conferences, like IEEE Vis and InfoVis, have NOT
%% in the past.

%% Please note that the use of figures other than the optional teaser is not permitted on the first page
%% of the journal version.  Figures should begin on the second page and be
%% in CMYK or Grey scale format, otherwise, colour shifting may occur
%% during the printing process.  Papers submitted with figures other than the optional teaser on the
%% first page will be refused. Also, the teaser figure should only have the
%% width of the abstract as the template enforces it.

%% These few lines make a distinction between latex and pdflatex calls and they
%% bring in essential packages for graphics and font handling.
%% Note that due to the \DeclareGraphicsExtensions{} call it is no longer necessary
%% to provide the the path and extension of a graphics file:
%% \includegraphics{diamondrule} is completely sufficient.
%%
\ifpdf%                                % if we use pdflatex
  \pdfoutput=1\relax                   % create PDFs from pdfLaTeX
  \pdfcompresslevel=9                  % PDF Compression
  \pdfoptionpdfminorversion=7          % create PDF 1.7
  \ExecuteOptions{pdftex}
  \usepackage{graphicx}                % allow us to embed graphics files
  \DeclareGraphicsExtensions{.pdf,.png,.jpg,.jpeg} % for pdflatex we expect .pdf, .png, or .jpg files
\else%                                 % else we use pure latex
  \ExecuteOptions{dvips}
  \usepackage{graphicx}                % allow us to embed graphics files
  \DeclareGraphicsExtensions{.eps}     % for pure latex we expect eps files
\fi%

%% it is recomended to use ``\autoref{sec:bla}'' instead of ``Fig.~\ref{sec:bla}''
\graphicspath{{figures/}{pictures/}{images/}{./}} % where to search for the images

\usepackage{microtype}                 % use micro-typography (slightly more compact, better to read)
\PassOptionsToPackage{warn}{textcomp}  % to address font issues with \textrightarrow
\usepackage{textcomp}                  % use better special symbols
\usepackage{mathptmx}                  % use matching math font
\usepackage{times}                     % we use Times as the main font
\renewcommand*\ttdefault{txtt}         % a nicer typewriter font
\usepackage{cite}                      % needed to automatically sort the references
\usepackage{color}
% \usepackage{tabu}                      % only used for the table example
% \usepackage{booktabs}                  % only used for the table example




\newcommand{\kallecomment}[1]{\textbf{[-Kalle-~}
    \textcolor{orange}{#1}
    \textbf{~]}}

\newcommand{\emilcomment}[1]{\textbf{[-Emil-~}
    \textcolor{red}{#1}
    \textbf{~]}}

\newcommand{\alexcomment}[1]{\textbf{[-Alex-~}
    \textcolor{magenta}{#1}
    \textbf{~]}}

\newcommand{\anderscomment}[1]{\textbf{[-Anders-~}
    \textcolor{cyan}{#1}
    \textbf{~]}}

\newcommand{\etal}{\emph{et~al.}}


%% We encourage the use of mathptmx for consistent usage of times font
%% throughout the proceedings. However, if you encounter conflicts
%% with other math-related packages, you may want to disable it.

%% In preprint mode you may define your own headline.
%\preprinttext{To appear in IEEE Transactions on Visualization and Computer Graphics.}

%% If you are submitting a paper to a conference for review with a double
%% blind reviewing process, please replace the value ``0'' below with your
%% OnlineID. Otherwise, you may safely leave it at ``0''.
\onlineid{0}

%% declare the category of your paper, only shown in review mode
\vgtccategory{Research}
%% please declare the paper type of your paper to help reviewers, only shown in review mode
%% choices:
%% * algorithm/technique
%% * application/design study
%% * evaluation
%% * system
%% * theory/model
\vgtcpapertype{System}

%% Paper title.
\title{Marsbrowsing}

%% This is how authors are specified in the journal style

%% indicate IEEE Member or Student Member in form indicated below
\author{Roy G. Biv, Ed Grimley, \textit{Member, IEEE}, and Martha Stewart}
\authorfooter{
%% insert punctuation at end of each item
\item
 Roy G. Biv is with Starbucks Research. E-mail: roy.g.biv@aol.com.
\item
 Ed Grimley is with Grimley Widgets, Inc.. E-mail: ed.grimley@aol.com.
\item
 Martha Stewart is with Martha Stewart Enterprises at Microsoft
 Research. E-mail: martha.stewart@marthastewart.com.
}

%other entries to be set up for journal
\shortauthortitle{Biv \MakeLowercase{\textit{et al.}}: Global Illumination for Fun and Profit}
%\shortauthortitle{Firstauthor \MakeLowercase{\textit{et al.}}: Paper Title}

%% Abstract section.
\abstract{
-
%
} % end of abstract

%% Keywords that describe your work. Will show as 'Index Terms' in journal
%% please capitalize first letter and insert punctuation after last keyword
\keywords{Radiosity, global illumination, constant time}

%% ACM Computing Classification System (CCS). 
%% See <http://www.acm.org/class/1998/> for details.
%% The ``\CCScat'' command takes four arguments.

\CCScatlist{ % not used in journal version
 \CCScat{K.6.1}{Management of Computing and Information Systems}%
{Project and People Management}{Life Cycle};
 \CCScat{K.7.m}{The Computing Profession}{Miscellaneous}{Ethics}
}

%% Uncomment below to include a teaser figure.
\teaser{
  \centering
  \includegraphics[width=\linewidth]{CypressView}
  \caption{In the Clouds: Vancouver from Cypress Mountain. Note that the teaser may not be wider than the abstract block.}
	\label{fig:teaser}
}

%% Uncomment below to disable the manuscript note
%\renewcommand{\manuscriptnotetxt}{}

%% Copyright space is enabled by default as required by guidelines.
%% It is disabled by the 'review' option or via the following command:
% \nocopyrightspace

\vgtcinsertpkg

%%%%%%%%%%%%%%%%%%%%%%%%%%%%%%%%%%%%%%%%%%%%%%%%%%%%%%%%%%%%%%%%
%%%%%%%%%%%%%%%%%%%%%% START OF THE PAPER %%%%%%%%%%%%%%%%%%%%%%
%%%%%%%%%%%%%%%%%%%%%%%%%%%%%%%%%%%%%%%%%%%%%%%%%%%%%%%%%%%%%%%%%

\begin{document}

%% The ``\maketitle'' command must be the first command after the
%% ``\begin{document}'' command. It prepares and prints the title block.

%% the only exception to this rule is the \firstsection command
\firstsection{Introduction} \label{sec:introduction}
\maketitle

\begin{enumerate}
\item Visualizing space data is important because its expensive
\item There exists a vast amount of data from Mars orbiters
\item Missing spatial understanding from looking at pure images
\item Contextualization of scientific data (being able to show satellites in the same context as surface features), (domes, vr headsets, etc)
\item Stereoscopic reconstruction from multiple image passes
\item What datasets are available (Viking, MOLA, CTX, HiRISE) and whats their resolutions
\item How can this be applied to other planets
\item A system for enabling future research thati s correctly contextualized
\item What is the science question // What is the point of this
\end{enumerate}
Length: About 1 page

\section{Related Work} \label{sec:relatedwork}
\begin{enumerate}
\item the book
\item terrain renderer
\item 3d reconstruction from images (stereoscopic and structure-from-motion)
\item GDAL
\item ``virtual presence'' systems
\item What else?
\end{enumerate}
Length: About 1 page\\
Note:  The page limit was increased to 9+2 pages this year (= 9 pages of manuscript, 2 pages of references). So we should make use of this and cite the hell out of everything that's related

\section{Overview} \label{sec:overview}
\begin{enumerate}
  \item What are the steps to get from a satellite to 3d terrain rendering
\begin{enumerate}
  \item Acquision (MRO information)
  \item Processing (AMES Stereo pipeline, ..., GDAL)
  \item Rendering (Globebrowsing)
\end{enumerate}
  \item short descriptions for each
\end{enumerate}
Length: About 2 pages

\section{Image Acquisition and Processing} \label{sec:imageacquisitionprocessing}
\begin{enumerate}
  \item MRO information, different resolution levels
  \item What are the available data products
  \item Ames stereo pipeline
  \item GDAL preprocessing
\end{enumerate}
Length: About 1-1.5 pages

\section{Rendering System} \label{sec:renderingsystem}
\begin{enumerate}
  \item All the steps to get from GDAL to a rendering on the screen
  \item Stereoscopic rendering
  \item Dome rendering
  \item Different resolution levels
  \item Rendering rover locations
\end{enumerate}
Length: About 2-2.5 pages (fill as much as the page limit (9+2) allows)

\section{Conclusion} \label{sec:system}
\begin{enumerate}
  \item Blabla; introduction in reverse
  \item Future work:
  \begin{enumerate}
    \item Focus more on scientific rather than engineering goals
  \end{enumerate}
\end{enumerate}
Length: About 1 page

%% if specified like this the section will be committed in review mode
\acknowledgments{
The authors wish to thank A, B, C. This work was supported in part by
a grant from XYZ.}

%\bibliographystyle{abbrv}
\bibliographystyle{abbrv-doi}
%\bibliographystyle{abbrv-doi-narrow}
%\bibliographystyle{abbrv-doi-hyperref}
%\bibliographystyle{abbrv-doi-hyperref-narrow}

\bibliography{references}
\end{document}

